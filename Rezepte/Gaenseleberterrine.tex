\begin{MyRecipe}{Gänseleberterrine}{ca. \Calc{5}{\x} Terrinen}{ca. einen Tag}

\textbf{Hinweis:}\\
Sollen die Pasteten rechtzeitig fertig sein, muss spätestens Vormittag der Vortages begonnen werden.\par\bigskip

\textbf{Vorbereitung Morgens (ca. \SI{2}{\hour}):}

\ingredient[\Calc{0.5}{\x}]{kg}{mag. Schweinebauch}
\ingredient[\Calc{1}{\x}]{kl.}{Zwiebel}
\ingredient[\Calc{2}{\x}]{}{Pimentkörner}
\ingredient[\Calc{1}{\x}]{}{Lorbeerblatt}
\ingredient[\Calc{1}{\x}]{TL}{Pfefferkörner}

Schweinebauch, Zwiebeln, Piment, Lorbeerblatt und Pfefferkörner mit kaltem Wasser abdecken und ca. \SI{2}{\hour} weich kochen.\par
Brühe durchseihen, das Fleisch in Fleischwolf-gerechte Stücke schneiden.\par\bigskip

\textbf{Stehen lassen (ca. \SI{3}{\hour}):}

\ingredient[\Calc{0.2}{\x}]{kg}{frischer Speck}
\ingredient[\Calc{0.5}{\x}]{kg}{Gänseleber}
\ingredient[]{}{Cognac}
Fleisch und Brühe gut abkühlen lassen, sodass Fett abgeschöpft werden kann.\par
Ein Drittel der Gänseleber in Cognac einlegen.\par
Frischen Speck würfeln und in Gefrierschrank legen.\par\bigskip

\textbf{Zubereitung Nachmittags (ca. \SI{1,5}{\hour}):}

\ingredient[\Calc{0.2}{\x}]{kg}{Kalbsleber}
\ingredient[\Calc{0.2}{\x}]{kg}{Schweinenüsschen}
\ingredient[\Calc{2}{\x}]{}{Zwiebeln}
\ingredient[\Calc{1}{\x}]{}{Knoblauchzehen}
\textit{Farce zubereiten:}\\	
Schweinebauch, Restl. Gänseleber, Kalbsleber und Nüsschen grob würfeln. Zwiebeln und Knoblauchzehen fein würfeln. Zutaten durch den Fleischwolf lassen. Zuletzt den angefrorenen frischen Speck durchlassen. Danach alles zusammen ein zweites mal durch den Fleischwolf drehen.

\ingredient[]{}{Weißwein}
\ingredient[]{}{Noilly Prat}
\ingredient[\Calc{3}{\x}]{}{Eigelb}
\ingredient[]{}{Pfeffer}
\ingredient[]{}{Salz}
\ingredient[\Calc{1}{\x}]{EL}{Thymian}
\ingredient[\Calc{1}{\x}]{EL}{Majoran}
\ingredient[\Calc{2}{\x}]{}{Pimentkörner}

\textit{Farce würzen:}\\	
Thymian, Majoran und Pimentkörner Mörsern und zusammen mit etwas Weißwein, etwas Noilly Prat, Eigelb, kräftig Salz und Pfeffer vermischen und rasch, aber gründlich mit Farce verrühren.



\ingredient[\Calc{2}{\x}]{}{Schalotten}
\ingredient[\Calc{6}{\x}]{Bl.}{Gelatine}
\ingredient[]{}{Brühe}
\ingredient[]{}{Weißwein}
\ingredient[]{}{Noilly Prat}
\ingredient[]{}{Portwein}
	
\textit{Jus:}\\	
Schalotten fein würfeln. Mit etwas Brühe, Weißwein, Noilly Prat und Protwein zu Jus reduzieren, durchseihen und abkühlen lassen.

\ingredient[]{}{Gänsefett}

\textit{Terrinen füllen}:\\
Terrinen mit Gänsefett einpinseln. Maximal zu etwa $\frac34$ füllen: Farce -- eingelegte Gänseleber -- Jus -- Farce\par\bigskip

\textbf{Backen (ca. 1.5~h):}\\
In Wasserbad bei \SI{175}{\celsius} Umluft, ca. 80'' backen.\par\bigskip

\textbf{Aspik (ca. 20''):}\\
Ausgetretenen Saft abgießen und mit Brühe zu \Calc{0.5}{1}~Liter auffüllen. Mit Gelatine nach Anleitung zu Aspik verarbeiten und wieder über die Terrinen gießen.\\
Über Nacht kühl stellen.
\par\bigskip



			
\end{MyRecipe}