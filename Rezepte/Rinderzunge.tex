\begin{MyRecipe}{Rinderzunge (Papa)}{\Calc{4}{\x} Personen}{}
	\ingredient[\Calc{1}{\x}]{} {Rinderzunge gepökelt}
	\ingredient[\Calc{1}{\x}]{} {Kalbszunge}
	\ingredient[\Calc{1}{\x}]{} {gr. Zwiebel}
	\ingredient[\Calc{1}{\x}]{} {kl. Lauch}
	\ingredient[\Calc{1}{\x}]{} {Petersilienwurzel}
	\ingredient[\Calc{1}{\x}]{} {Knoblauchzehe}
	\ingredient[\Calc{2}{\x}]{} {Karotten}
	\ingredient[\Calc{0.5}{\x}]{} {kl. Sellerie}
	\Step{Vorbereiten}
	Zwiebeln von äußerster schale befreien und halbieren. Restliches Gemüse waschen, und grob würfeln.

	Zungen gründlich waschen und abschaben.\par\bigskip
	
	\ingredient[\Calc{1}{\x}]{} {Lorbeerblatt}
	\ingredient[\Calc{1}{\x}]{TL} {Pfefferkörner}
	\ingredient[]{etwas} {Salz}
	\Step{Brühe ansetzen}
	Zwiebeln mit etwas Öl anbraten. Zwiebeln mit etwas Wasser ablöschen, dann die Zungen und das restliche Gemüse einbetten und mit kaltem Wasser auffüllen, sodass alles bedeckt ist.
	Gewürze hinzugeben.\par\bigskip
	
	\Step{Köcheln lassen}
	Ca. \SI{1}{\hour} bis \SI{2}{\hour} simmern lassen und die Kalbszunge herausnehmen, sobald diese weich ist. Sofort die heiße Zunge häuten. Nach ca. weiteren \SI{1}{\hour} bis \SI{2}{\hour} die Rinderzunge herausnehmen, sobald diese weich ist und ebenfalls noch im heißen Zustand häuten.

	Restliche Brühe durch ein Tuch geben.\par\bigskip

	\Step{Das passt dazu}
	Spargel, Salzkartoffeln, Flädle und etwas flüssige Butter
	

	
	
	
\end{MyRecipe}