\begin{MyRecipe}{Rinderbrühe}{\Calc{1}{\x} Topf voll}{15'' + 90''}

\ingredient[\Calc{1}{\x}]{} {kl. Lauch}
\ingredient[\Calc{2}{\x}]{} {gr. Zwiebeln}
\ingredient[\Calc{1}{\x}]{} {kl. Sellerie}
\ingredient[\Calc{3}{\x}]{} {gr. Möhren}
\ingredient[\Calc{0.5}{\x}]{Bd} {Petersilie}
\ingredient[\Calc{2}{\x}]{} {Knoblauchzehen}
\textbf{Vorbereiten:}\\
Gemüse waschen, schälen und in große Stücke würfeln.

\ingredient[\Calc{0.4}{\x}]{g} {krause Knochen (Sandknochen)}
\ingredient[\Calc{0.4}{\x}]{g} {Markknochen}
\textbf{Anbraten:}\\
Zunächst die Knochen in Öl kräftig anbraten. Dann das Gemüse hinzufügen.

\ingredient[\Calc{0.7}{\x}]{g} {Wade / Brustkern}
\ingredient[\Calc{2}{\x}]{l} {Wasser}
\ingredient[1]{} {Lorbeerblatt}
\ingredient[]{etw.} {Liebstöckel}
\ingredient[\Calc{1}{\x}]{} {Nelke}
\ingredient[\Calc{1}{\x}]{TL} {Salz}
\ingredient[\Calc{1}{\x}]{TL} {Pfefferkörner}
\ingredient[\Calc{1}{\x}]{TL} {Senfkörner}

\textbf{Fleisch einbetten:}\\
Fleisch auf das Gemüse legen und mit kaltem Wasser übergießen, bis alles unter Wasser liegt. Gewürze entweder in großzügigem Teesieb einhängen oder direkt in die Suppe geben.

\textbf{Köcheln lassen:}\\
\SI{1,5}{\hour} leise köcheln lassen und anfangs etwas den entstehenden Schaum abschöpfen. Sobald die Wade weich ist, vom Herd nehmen.\par

\ingredient[]{} {Salz}
\ingredient[]{} {Pfeffer}
\ingredient[]{} {Zucker}
\textbf{Abschmecken:}\\
Am nächsten Tag, Gemüse und Fleisch von der Brühe trennen, Fett abschöpfen und mit Salz, Pfeffer und Zucker abschmecken.









\end{MyRecipe}