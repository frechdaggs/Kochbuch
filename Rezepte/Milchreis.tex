\begin{MyRecipe}{Milchreis}{\Calc{4}{\x} Personen}{\SI{10}{\minuteprime} + \SI{30}{\minuteprime}}
	
	\ingredient[\Calc{0.25}{\x}]{\si{\kilogram}} {Milchreis}
	\Step{Vorbereitung}
	Milchreis gut waschen und abtropfen lassen.\par\bigskip

	\ingredient[\Calc{1}{\x}]{\si{\liter}} {Milch}
	\ingredient[\Calc{1}{\x}]{\si{\Essloeffel}} {Butter}
	\ingredient[\Calc{4}{\x}]{\si{\Essloeffel}} {Zucker}
	\ingredient[\Calc{1}{\x}]{} {Vanilleschote}
	
	\Step{Ansetzen}
	In einem Topf mit möglichst großer Bodenfläche die Butter schmelzen. Dann den Milchreis kurz darin andünsten und die (zimmerwarme) Milch hinzugeben. 
	
	Vanilleschote aufschlitzen und das Mark der Vanilleschote sowie die aufgeschlitzte Vanilleschote selbst zusammen mit dem Zucker hinzugeben. Unter ständigem Rühren einmal kräftig aufkochen lassen und dabei aufpassen,
	dass sich nichts am Topfboden ansetzt.\par\bigskip
	
	\Step{Ziehen lassen}
	Nun die aufgeschlitzte Vanilleschote herausfischen, die Temperatur der Herdplatte zurückschalten, sodass die Milch gerade noch leicht köchelt und Topf mit Deckel verschließen.
	
	Nach ca. \SI{15}{\minuteprime} (evtl. auch früher) einmal ordentlich umrühren und die Temperatur am Herd gegebenenfalls anpassen.
	
	Nach weiteren \SI{15}{\minuteprime} ist der Milchreis fertig.\par\bigskip
	
	\ingredient[]{} {Zimt}
	\ingredient[]{} {Zucker}
	\ingredient[]{} {Apfelmus}
	\ingredient[]{} {Schattenmorellen}
	\Step{Servieren}
	Serviert werden kann der Milchreis mit Schattenmorellen, Zimt \& Zucker, Apfelmus oder allem zusammen.\par\bigskip
		
	\Step{Tipp}
	Sollte der Topfdeckel nicht ausreichend schießen, kann ein Geschirrtuch als Dichtung zwischen Topf und Deckel geklemmt werden.
	
	
	
	
	
	
	
\end{MyRecipe}