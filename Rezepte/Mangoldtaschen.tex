\begin{MyRecipe}{Mangoldtaschen}{\Calc{2}{\x} Personen (\Calc{6}{\x} Stück)}{20'' + 20''}
	
	\ingredient[\Calc{6}{\x}]{} {gr. Blätter Mangold (evt. noch kleine dazu)}
	\ingredient[\Calc{6}{\x}]{} {Pfitzauf-Formen}
	\Step{Vorbereitung}
	Mangold gut waschen und von Schadstellen befreien. Aus den großen Blättern den harten Strunk vorsichtig heraus schneiden. Die Abschnitte vom Mangold \emph{nicht} wegschmeißen.
	
	Pfitzauf-Formen mit Olivenöl ausstreichen und mit den großen Mangold-Blättern auslegen.\par\bigskip
	
	\ingredient[\Calc{1}{\x}]{} {Zwiebeln}
	\ingredient[\Calc{2}{\x}]{} {Knoblauchzehen}
	\ingredient[\Calc{0.5}{\x}]{Bnd.} {Petersilie}
	\ingredient[]{etw.} {Butter}
	\ingredient[\Calc{0.2}{\x}]{kg} {Ziegenkäse-Rolle}
	\ingredient[\Calc{1}{\x}]{} {Eier}
	\ingredient[\Calc{1}{\x}]{} {Semmelbrösel}
	\ingredient[]{etw.} {Salz}
	\ingredient[]{etw.} {Pfeffer}
	\ingredient[]{etw.} {Olivenöl}
	\Step{Füllung}
	Die Abschnitte vom Mangold, die kleineren Blätter, die Zwiebeln sowie die Knoblauchzehen\footnote{evt. Walnüsse?} fein würfeln und in Butter an schwitzen.

	Etwas abkühlen lassen.
	
	Währenddessen Käse-Rolle in grobe Scheiben schneiden und zusammen mit den Eiern zu einer nicht ganz homogenen Masse zerdrücken.
	
	Zwiebel-Mangold-Schwitze zusammen mit etwas Semmelbrösel (die Masse sollte nicht trocken/fest werden), Salz und Pfeffer in die Ei-Käse-Mischung einarbeiten und vermischen.

	Füllung in die mit den großen Mangold-Blättern ausgelegten Pfitzauf-Formen verteilen. Anschließend die Mangold-Blätter über der Füllung zusammenschlagen und mit ordentlich Olivenöl übergießen. Zuletzt mit etwas Semmelbrösel bestreuen.
	\par\bigskip

	\Step{Backen}
	Bei \SI{200}{\degreeCelsius} circa \SI{20}{\minute} bei Ober-/Unterhitze backen. Der Mangold sollte an der Oberfläche schön knusprig aber nicht schwarz werden.
	\par\bigskip

	\Step{Das passt dazu}
	Pellkartoffeln, Ravioli, Butter-Nudeln
	

	
	
\end{MyRecipe}