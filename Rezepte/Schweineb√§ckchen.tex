\begin{MyRecipe}{Geschmorte Schweinebäckchen}{\Calc{4}{\x} Personen}{\SI{45}{\minuteprime} + \SI{2,5}{\hour}}
	
	\ingredient[\Calc{1}{\x}]{} {gr. Zwiebel}
	\ingredient[\Calc{2}{\x}]{} {gr. Karotten}
	\ingredient[\Calc{0.5}{\x}]{} {Knollensellerie}
	\ingredient[\Calc{1}{\x}]{} {kl. Lauch}
	\Step{Vorbereitung}
	Zwiebeln von äußerster Schale befreien und halbieren.
	
	Karotten, Knollensellerie waschen, schälen und in grobe Würfel scheiden (ca. \SI{1}{\centi\meter}).
	
	Den Lauch gründlich waschen und in grobe Ringe schneiden  (ca. \SI{1}{\centi\meter}).\par\bigskip
	
	\ingredient[\Calc{4}{\x}]{\si{\Essloeffel}} {Olivenöl}
	\ingredient[\Calc{40}{\x}]{\si{\gram}} {Tomatenmark}
	\ingredient[\Calc{0.25}{\x}]{\si{\liter}} {Rotwein}
	\ingredient[\Calc{4}{\x}]{} {Knoblauchzehen}
	\ingredient[\Calc{1}{\x}]{} {Lorbeerblätter}
	\ingredient[\Calc{3}{\x}]{} {Thymianzweige}
	\Step{Ansetzen}
	In einem Schmortopf ordentlich Öl erhitzen und die Zwiebeln auf ihrer Schnittfläche kurz anbraten. 
	
	Dann das restliche Gemüse hinzugeben und alles zusammen nochmals kräftig anbraten.

	Anschließend das Tomatenmark hinzugeben und alles bei schwacher Hitze einige Minuten rösten. 
	
	Mit dem Rotwein nach und nach das ganze ablöschen und immer wieder etwas einkochen lassen.\par\bigskip
	
	\ingredient[\Calc{12}{\x}]{} {Schweinebäckchen}
	\ingredient[\Calc{4}{\x}]{} {Knoblauchzehen}
	\ingredient[\Calc{1}{\x}]{} {Lorbeerblätter}
	\ingredient[\Calc{3}{\x}]{} {Thymianzweige}
	\ingredient[]{} {Salz}
	\ingredient[]{} {Pfeffer}
	\Step{Schmoren}
	Dann Knoblauchzehen, Lorbeerblätter und Thymianzweige sowie etwas Salz und Pfeffer hinzufügen, mit Wasser bedecken und einmal aufkochen lassen.
	
	Dann die Schweinebäckchen einbetten und mit geschlossenem Deckel ca. \SIrange{1}{2}{\hour} bei schwacher Hitze weich schmoren.\par\bigskip
	
	\ingredient[\Calc{50}{\x}]{\si{\milli\liter}} {Sahne}
	\Step{Sauce}
	Die Schweinebäckchen herausnehmen und warmstellen.
	
	Die Sauce durch ein feines Sieb geben. Je nach gewünschter Konsistenz kann anschließend das Gemüse passiert und die Sauce eingekocht werden. 
	
	Anschließend die Sauce mit der Sahne abziehen und mit Salz und Pfeffer abschmecken.\par\bigskip
	
	\Step{Servieren}
	Sehr gut passt Kartoffel- oder Selleriepüree dazu. Oder einfach klassisch mit Spätzle, Kartoffelgratin oder einfachen Bandnudeln.\par\bigskip

	\Step{Tipp}
	Mit Vorspeise und Nachtisch reichen 2 Bäckchen pro Person.
	
\end{MyRecipe}