\begin{MyRecipe}{Kartoffel-Gemüse-Gratin}{\Calc{4}{\x} Personen}{\SI{30}{\minuteprime} + \SI{35}{\minuteprime}}

	\ingredient[\Calc{0.5}{\x}]{\si{\kilogram}} {Kartoffeln, (vorw.) festkochend}
	\ingredient[\Calc{0.5}{\x}]{\si{\kilogram}} {anderes Gemüse}
	\ingredient[\Calc{1}{\x}]{} {Zwiebeln}
	\ingredient[\Calc{0.15}{\x}]{\si{\kilogram}} {Schinken}	\ingredient[\Calc{0.25}{\x}]{\si{\liter}} {Milch}
	\ingredient[\Calc{0.25}{\x}]{\si{\liter}} {Sahne}
	\ingredient[]{etw.} {Gemüsebrühe}
	\ingredient[]{etw.} {Salz}
	\ingredient[]{etw.} {Pfeffer}
	\ingredient[]{etw.} {Muskat}
	
	\Step{Vorbereitung}
	Die Kartoffeln und anderes Gemüse (Karotten, Kohlrabi, gelbe Beete, Blumenkohl, Brokkoli...) schälen und in dünne Scheiben schneiden.
	
	Zwiebeln schälen und fein hacken.
	
	Schinken in feine Streifen oder Würfel schneiden. (Kann für die Vegetarier auch weggelassen werden.)
	
	Milch und Sahne (Verhältnis kann den Umständen im Kühlschrank angepasst werden) in einem Messbecher vermischen. Mit Gemüsebrühe, Salz, Pfeffer und Muskat würzen.\par\bigskip
	

	\ingredient[\Calc{50}{\x}]{\si{\gram}} {Butter}
	\ingredient[\Calc{2}{\x}]{EL, geh.} {Mehl}
	\Step{Sauce}
	Butter in großem Topf erhitzen und Zwiebeln zusammen mit dem Schinken darin glasig dünsten. Anschließend mit Mehl bestäuben und weitere \SIrange{1}{2}{\minuteprime} an schwitzen. Nun nach und nach die Sahne-Milch-Mischung unter ständigem Rühren dazugeben. Sobald die komplette Sauce angerührt ist, das Gemüse dazugeben und nochmals aufkochen lassen. Dabei aufpassen, dass nichts anbrennt.\par\bigskip
	
	\ingredient[\Calc{0.2}{\x}]{\si{\kilogram}} {Gouda}
	\ingredient[\Calc{0.1}{\x}]{\si{\kilogram}} {Parmesan}
	\ingredient[]{etw.} {Butter}	
	\Step{In Auflaufform geben}
	Auflaufform mit Butter ausstreichen und die Gemüse-Sauce-Mischung hineingeben. Dann den Gouda und abschließend den Parmesan drüber reiben.\par\bigskip
	
	\Step{Backen}
	Bei \SI{180}{\degreeCelsius} Ober-/Unterhitze etwa \SI{60}{\minuteprime} backen bis die Kartoffeln sich leicht durchstechen lassen.
		
\end{MyRecipe}