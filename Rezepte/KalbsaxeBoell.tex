\begin{MyRecipe}{Kalbshaxe vom Böll}{für \Calc{2}{\x} Haxen}{30'' + 120''}
	\ingredient[\Calc{2}{\x}]{}{Kalbshaxen}
	\ingredient[]{}{Kümmel, ganz}
	\ingredient[viel]{}{Senf}
	\ingredient[\Calc{2}{\x}]{}{gr. Möhren}
	\ingredient[\Calc{1}{\x}]{}{Petersilienwurzel}
	\ingredient[\Calc{0.5}{\x}]{}{Sellerie}
	\ingredient[\Calc{1}{\x}]{}{Lauch}
	\ingredient[\Calc{2}{\x}]{}{gr. Zwiebeln}
	\ingredient[]{}{Blattpetersilie}
	\ingredient[]{}{frischer Majoran}
	\ingredient[]{}{frischer Thymian}
	\ingredient[]{}{Piment}
	\ingredient[]{}{Wacholder}
	\ingredient[]{}{Lorbeer}
	
	\textbf{Gemüse braten:}\\
	Gemüse grob kleinschneiden. Knochen im Bräter in Gänseschmalz anbraten, Gemüse dazu und auch kurz mit anbraten. Alles raus und beiseite.\par\bigskip

	\textbf{Haxen braten:}\\
	Kalbshaxen salzen und mit Senf und Kümmel ordentlich einreiben. Von allen Seiten im Bräter mit Gänseschmalz scharf anbraten.\par\bigskip

	\textbf{In den Ofen:}\\
	Wasser langsam angießen. Gemüse und Knochen in den Bräter und Kräuter dazu, pfeffern, Lorbeer, Piment, Wacholder. Mindestens 2 Stunden bei 200°C im Ofen. Dabei immer wieder wenden und
	heißes Wasser nachgießen.\par\bigskip
	
	\textbf{Soße:}\\
	Wenn die Haxen gar sind, mit heißen Wasser aus dem Gemüse und dem Bodensatz Soße herstellen.\par\bigskip

	\textbf{Das passt dazu:}\\
	Kartoffelklöße, Rotkohl, Wirsing.
		

				
	\end{MyRecipe}