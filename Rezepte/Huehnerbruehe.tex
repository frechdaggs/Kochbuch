\begin{MyRecipe}{Hühnerbrühe}{\Calc{1}{\x} Suppenhuhn}{\SI{15}{\minuteprime}+\SI{90}{\minuteprime}}

\ingredient[\Calc{2}{\x}]{} {gr. Möhren}
\ingredient[\Calc{1}{\x}]{} {kl. Lauch}
\ingredient[\Calc{2}{\x}]{} {gr. Zwiebeln}
\ingredient[\Calc{1}{\x}]{} {kl. Sellerie}
\ingredient[\Calc{3}{\x}]{} {gr. Möhren}
\ingredient[\Calc{0.5}{\x}]{Bd} {Petersilie}
\ingredient[\Calc{2}{\x}]{} {Knoblauchzehen}
\textbf{Gemüse anbraten:}\\
Zwiebeln nur die äußerste Schale entfernen, halbieren und mit der Schnittseite in Öl anbraten. Restliches Gemüse waschen, schälen, in große Stücke würfeln und dann zu den Zwiebeln in den Topf geben.

\ingredient[\Calc{1}{\x}]{} {Suppenhuhn (\SI{1,6}{\kilogram})}
\ingredient[\Calc{3}{\x}]{l} {Wasser}
\ingredient[\Calc{2}{\x}]{} {Lorbeerblatt}
\ingredient[\Calc{4}{\x}]{} {Thymianstiele}
\ingredient[\Calc{2}{\x}]{} {Nelken}
\ingredient[\Calc{1}{\x}]{TL} {Salz}
\ingredient[\Calc{2}{\x}]{TL} {Pfefferkörner}

\textbf{Huhn einbetten:}\\
Suppenhuhn auf das Gemüse legen und mit kaltem Wasser übergießen, bis alles unter Wasser liegt. Gewürze entweder in großzügigem Teesieb einhängen oder direkt in die Suppe geben.

\textbf{Köcheln lassen:}\\
\SI{1,5}{\hour} leise köcheln lassen und anfangs etwas den entstehenden Schaum abschöpfen. Sobald die sich die Keulen leicht vom Suppenhuhn ziehen lassen, vom Herd nehmen.\par

\ingredient[]{} {Salz}
\ingredient[]{} {Pfeffer}
\ingredient[]{} {Zucker}
\textbf{Abschmecken:}\\
Am nächsten Tag, Gemüse und Fleisch von der Brühe trennen, Fett abschöpfen und mit Salz, Pfeffer und Zucker abschmecken.









\end{MyRecipe}