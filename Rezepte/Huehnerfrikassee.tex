\begin{MyRecipe}{Hühnerfrikassee}{\Calc{4}{\x} Personen}{30''}
\ingredient[\Calc{0.1}{\x}]{\si{\liter}} {Weißwein}
\ingredient[\Calc{0.3}{\x}]{\si{\liter}} {kräftige Hühnerbrühe}
\ingredient[\Calc{25}{\x}]{\si{\gram}} {Butter}
\ingredient[\Calc{25}{\x}]{\si{\gram}} {Mehl}
\textbf{Mehlschwitze:}\\
Mehl in Butter anschwitzen und dann bei ständigem Rühren langsam zuerst den Wein, dann die Brühe hinzugeben.\par\bigskip

\ingredient[\Calc{0.1}{\x}]{\si{\milli\liter}} {Sahne}
\ingredient[\Calc{0.15}{\x}]{\si{\gram}} {Möhrchen aus dem Glas}
\ingredient[\Calc{0.15}{\x}]{\si{\gram}} {TK-Erbsen}
\textbf{Gemüse dazu:}\\
Mehlschwitze mit Sahne abziehen und Gemüse hinzu geben. \SI{5}{\minute} köcheln lassen\par\bigskip.

\ingredient[]{etw.} {Zitronensaft}
\ingredient[]{} {Salz}
\ingredient[]{} {Pfeffer}
\textbf{Abschmecken:}\\
Mit einem Spritzer Zitrone, Salz und Pfeffer abschmecken.\par\bigskip

\ingredient[]{} {gek. Suppenhuhn}
\textbf{Fleisch vom Suppenhuhn hinzugeben:}\\
Vom Suppenhuhn abgezupftes Fleisch evtl. etwas klein schneiden und mit ins Frikassee geben. \SI{5}{\minute} ziehen lassen.\par\bigskip

\textbf{Das passt dazu:}\\
Reis

\end{MyRecipe}