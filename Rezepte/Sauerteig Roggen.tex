\begin{MyRecipe}{Sauerteig --- Roggen}{}{6 Tage à 10''}
	
	\ingredient[\Calc{25}{\x}]{\si{\gram}}{Roggenvollkornmehl}
	\ingredient[\Calc{25}{\x}]{\si{\gram}}{lauwarmes Wasser}
	
	\Step{Tag 1:}
	Roggenvollkornmehl mit Wasser zu einer homogenen Masse vermengen und \si{24}{\hour} bei \si{25}{\degreeCelsius} lagern.
	
	\ingredient[]{}{Teig von Tag 1}
	\ingredient[\Calc{25}{\x}]{\si{\gram}}{Roggenvollkornmehl}
	\ingredient[\Calc{25}{\x}]{\si{\gram}}{lauwarmes Wasser}
	
	\Step{Tag 2:}
	Den gesamten Teig von Tag 1, Roggenvollkornmehl mit Wasser zu einer homogenen Masse vermengen und \si{24}{\hour} bei \si{25}{\degreeCelsius} lagern.
	
	\ingredient[\Calc{25}{\x}]{\si{\gram}}{Teig vom Vortag}
	\ingredient[\Calc{50}{\x}]{\si{\gram}}{Roggenvollkornmehl}
	\ingredient[\Calc{50}{\x}]{\si{\gram}}{lauwarmes Wasser}
	
	\Step{Tag 3 - 5:}
	Einen Teil des Teiges vom Vortag, Roggenvollkornmehl mit Wasser zu einer homogenen Masse vermengen und \si{24}{\hour} bei \si{25}{\degreeCelsius} in einem sauberen Glas lagern.
	
	\Step{Tag 6:}
	Der Sauerteig sollte nun über nach etwa das doppelte seines Volumens angenommen haben und ist nun bereit zum Backen.
	
	\ingredient[\Calc{25}{\x}]{\si{\gram}}{Anstellgut}
	\ingredient[\Calc{50}{\x}]{\si{\gram}}{Roggenvollkornmehl}
	\ingredient[\Calc{50}{\x}]{\si{\gram}}{lauwarmes Wasser}

	\Step{Füttern:}
	Einen Teil des Anstellguts, Roggenvollkornmehl mit Wasser zu einer homogenen Masse vermengen sauberen Glas lagern. Sobald sich das Volumen verdoppelt hat, kann das neue Anstellgut in den Kühlschrank
	
	\Step{Tipps:}
	\begin{itemize}
		\item Gefüttert werden sollte alle 2 Wochen oder spätestens wenn sich ein Fusel bildet.
		\item Um die Volumenvergrößerung messen zu können, kann ein Gummi um das Glas zur Markierung des Füllstandes verwendet werden.
		\item Bei einem jungen Sauerteil sollte in den Brotteig \si{1-2}{\gram} hinzugegeben werden.
	\end{itemize}
	
	Zwiebeln, Möhren und Staudensellerie putzen bzw. schälen, ganz fein hacken. Einen ordentlichen Klotz Butter und ein wenig Olivenöl (damit die Butter nicht verbrennt) in einen großen Topf geben. Gehacktes Gemüse in der Butter auf möglichst niedriger Flamme langsam (mind. \SI{30}{\minute}) garen. Wichtig ist, dass das Gemüse nicht brät, sondern wirklich nur dünstet --- also nicht zu heiß werden lassen!\par\bigskip
	
	\ingredient[\Calc{1}{\x}]{\si{\kilogram}}{gem. Hackfleisch}
	\ingredient[etw.]{}{Weißwein}
	\ingredient[\Calc{0.4}{\x}]{\si{\liter}}{Milch}
	
	\Step{Hackfleisch braten}
	Während das Soffritto gart, das Rinderhackfleisch portionsweise so lange in einer weiteren Pfanne scharf braten, bis es kräftig Farbe annimmt. Jede Portion  anschließend mit einem Schuss Weißwein ablöschen, sodass sich alles Angebackene vom Boden lösen lässt. Sobald das Hackfleisch komplett angebraten ist, das gesamte Hackfleisch wieder zurück in die Pfanne geben und mit der Milch bedecken und aufkochen lassen.\par\bigskip
	
	\ingredient[\Calc{0.6}{\x}]{\si{\kilogram}}{Dosentomaten (San Marzano)}
	\ingredient[\Calc{3}{\x}]{Zehen}{Knoblauch}
	\ingredient[etw.]{}{Harissa}
	\ingredient[]{}{Salz}
	\ingredient[]{}{Pfeffer}
	
	\Step{Sugo}
	Sobald das Soffritto gar ist und das Hackfleisch mit der Milch zusammen aufgekocht ist, die Tomaten und die Hackfleischmilch zu dem Soffritto hinzugeben. Den in feine Scheiben geschnittene Knoblauch sowie einen Tropfen Harissa, Salz und Pfeffer in die Sauße geben. Nun die Sauße mindestens \SI{2}{\hour}, besser \SI{5}{\hour} (je länger desto besser) köcheln lassen und von Zeit zu Zeit umrühren und ggf. etwas Wasser nachgeben.
	
	\ingredient[\Calc{0.6}{\x}]{kg}{Spaghetti}
	\ingredient[]{}{Parmesan}
	\ingredient[]{}{Gouda}
	
	\Step{Außerdem}
	
	
\end{MyRecipe}