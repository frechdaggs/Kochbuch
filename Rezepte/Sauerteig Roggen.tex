\begin{MyRecipe}{Sauerteig --- Roggen}{}{6 Tage à \SI{10}{\minuteprime}}
	
	\ingredient[\Calc{25}{\x}]{\si{\gram}}{Roggenvollkornmehl}
	\ingredient[\Calc{25}{\x}]{\si{\gram}}{lauwarmes Wasser}
	
	\Step{Tag 1}
	Roggenvollkornmehl mit Wasser zu einer homogenen Masse vermengen und \SI{24}{\hour} bei \SI{25}{\degreeCelsius} lagern.\par\bigskip
	
	\ingredient[]{}{Teig von Tag 1}
	\ingredient[\Calc{25}{\x}]{\si{\gram}}{Roggenvollkornmehl}
	\ingredient[\Calc{25}{\x}]{\si{\gram}}{lauwarmes Wasser}
	
	\Step{Tag 2}
	Den gesamten Teig von Tag 1, Roggenvollkornmehl mit Wasser zu einer homogenen Masse vermengen und \SI{24}{\hour} bei \SI{25}{\degreeCelsius} lagern.\par\bigskip
	
	\ingredient[\Calc{25}{\x}]{\si{\gram}}{Teig vom Vortag}
	\ingredient[\Calc{50}{\x}]{\si{\gram}}{Roggenvollkornmehl}
	\ingredient[\Calc{50}{\x}]{\si{\gram}}{lauwarmes Wasser}
	
	\Step{Tag 3 - 5}
	Einen Teil des Teiges vom Vortag, Roggenvollkornmehl mit Wasser zu einer homogenen Masse vermengen und \SI{24}{\hour} bei \SI{25}{\degreeCelsius} in einem sauberen Glas lagern.\par\bigskip
	
	\Step{Tag 6:}
	Der Sauerteig sollte nun über nach etwa das doppelte seines Volumens angenommen haben und ist nun bereit zum Backen.\par\bigskip
	
	\ingredient[\Calc{25}{\x}]{\si{\gram}}{Anstellgut}
	\ingredient[\Calc{50}{\x}]{\si{\gram}}{Roggenvollkornmehl}
	\ingredient[\Calc{50}{\x}]{\si{\gram}}{lauwarmes Wasser}

	\Step{Füttern}
	Einen Teil des Anstellguts, Roggenvollkornmehl mit Wasser zu einer homogenen Masse vermengen sauberen Glas lagern. Sobald sich das Volumen verdoppelt hat, kann das neue Anstellgut in den Kühlschrank\par\bigskip
	
	\Step{Tipps}
	\begin{itemize}
		\item Gefüttert werden sollte alle 2 Wochen oder spätestens wenn sich ein Fusel bildet.
		\item Um die Volumenvergrößerung messen zu können, kann ein Gummi um das Glas zur Markierung des Füllstandes verwendet werden.
		\item Bei einem jungen Sauerteil sollte in den Brotteig \SI{2}{\gram} Hefe hinzugegeben werden.
	\end{itemize}
	
	
	
\end{MyRecipe}