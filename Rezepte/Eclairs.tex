\begin{MyRecipe}{Eclairs}{\Calc{14}{\x} Stück}{}

	\ingredient[\Calc{4}{\x}]{}{Eigelb}
	\ingredient[\Calc{0.1}{\x}]{\si{\kilogram}}{Zucker}
	\ingredient[\Calc{40}{\x}]{\si{\gram}}{Speisestärke}
	\ingredient[\Calc{0.5}{\x}]{\si{\liter}}{Milch}
	\ingredient[\Calc{1}{\x}]{Schote}{Vanille}
	\ingredient[\Calc{1}{\x}]{Pr.}{Salz}
	
	\Step{Füllung}
	Eigelb, Stärke, Zucker, Salz verrühren. Die Milch zusammen mit dem Vanillemark und der ausgekratzten Vanilleschote aufkochen. Sobald die Milch kocht diese Schluckweise unter die Eigelb-Masse rühren um diese zu temperieren. Wenn etwa die Hälfte der Milch untergerührt wurde, die Eimasse zurück in den Topf geben und bei mittlerer Hitze zum köcheln bringen bis der Pudding beginnt fest zu werden und kocht.\\
	Pudding saubere Schüssel abfüllen und mit Frischhaltefolie abdecken. Erst bei Raumtemperatur, dann im Kühlschrank mind. \SI{4}{\hour} komplett abkühlen.\par\bigskip

	
	\ingredient[\Calc{25}{\x}]{\si{\centi\liter}}{Wasser}
	\ingredient[\Calc{50}{\x}]{\si{\gram}}{Butter}
	\ingredient[\Calc{150}{\x}]{\si{\gram}}{Mehl}
	\ingredient[\Calc{4}{\x}]{}{Eier}
	\ingredient[\Calc{1}{\x}]{Pr.}{Salz}
	
	\Step{Brandteig}
	Backofen auf \SI{200}{\degreeCelsius} Ober- und Unterhitze vorheizen.\\
	Wasser, Butter und Salz in einem Topf zum Kochen bringen. Mehl hinein stürzen und mit Kochlöffel so lange kräftig umrühren, bis ein glatter Teigklumpen entsteht und sich am Boden des Topfes eine weiße Schicht bildet\footnote{Diesen Schritt nennt man Abbrennen des Teiges.}.\\
	Teig etwas abkühlen lassen. Nach und nach mit den Eiern verrühren und auf ein Backblech ca. 5 cm lang spritzen.\\
	Backen\\
	Abkühlen\par\bigskip
	
	\Step{Backen}
	Bei \SI{200}{\degreeCelsius} Ober- und Unterhitze etwa \SI{25}{\minute} backen bis Eclairs goldbraun gebacken sind. Nach dem Backen komplett abkühlen lassen.\par\bigskip
		
	\ingredient[\Calc{250}{\x}]{\si{\gram}}{Sahne}
	
	\Step{Füllen}
	Für die Füllung müssen die Eclairs komplett abkühlen. Pudding glatt rühren und mit steif geschlagener Sahne unterheben. Masse in einen Spritzbeutel mit kleiner oder großer Sterntülle füllen. Eclairs wahlweise mit dem Messer aufschneiden oder zwei Löcher in die Unterseite jedes Eclairs pieksen und in die Eclairs mit der Creme befüllen.\par\bigskip
	
	\ingredient[\Calc{150}{\x}]{\si{\gram}}{Zartbitterschokolade}
	\ingredient[\Calc{70}{\x}]{\si{\gram}}{Sahne}
	\ingredient[\Calc{10}{\x}]{\si{\milli\liter}}{Congnac (optional)}
	
	\Step{Glasur}
	Für die Glasur Schokolade, Sahne und Cognac in eine kleinen Schüssel über dem Wasserbad schmelzen. Glatt rühren und die Eclairs darin eintauchen. Bis zum Servieren in den Kühlschrank stellen.\par\bigskip
	
	\url{https://heissehimbeeren.com/franzoesische-eclairs-mit-vanillecreme-fuellung-originalrezept/}
	
	\Step{Fazit}
	\begin{itemize}
		\item Glasur sollte mehr dem Guss von Schokoladen-Berlinern ähneln. 
		\item Die Füllung sollte ohne Schlagsahne sein.
		\item beim Backen die Eclairs als einzelne Stränge spritzen
	\end{itemize}

	
	
\end{MyRecipe}