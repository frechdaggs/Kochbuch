\begin{MyRecipe}{Hackfleischbällchen m. Kohlrabi-Sahnesoße}{\Calc{4}{\x} Personen}{30''}
	
	\ingredient[\Calc{0.5}{\x}]{kg}{gem. Hackfleisch}
	\ingredient[\Calc{1}{\x}]{}{kl. Ziebel}
	\ingredient[\Calc{1}{\x}]{}{Ei}
	\ingredient[\Calc{1}{\x}]{EL}{Semmelbrösel}
	\ingredient[]{}{Salz}
	\ingredient[]{}{Pfeffer}
	\ingredient[]{}{Senf}
	
	\textbf{Hackfleischteig vorbereiten:}\\
	Hackfleisch, sehr fein geschnittene Zwiebel, Ei, Semmelbrösel mit etw. Senf, Salz und Pfeffer gut durchmischen.\par\bigskip
	
	\ingredient[\Calc{3}{\x}]{}{mittlere Kohlrabi mit Blättern}
	\ingredient[\Calc{0.5}{\x}]{Bund}{Petersilie}
	\ingredient[]{}{Butter}

	\textbf{Kohlrabi zubereiten:}\\
	Kohlrabi schälen, holzige Stellen entfernen, vierteln und in Scheiben schneiden. Junge Kohlrabiblätter vom Stiel befreien und zusammen mit Pertersilie hacken. Kohlrabischeiben mit Grünzeug in großer Pfanne mit etw. Butter und Wasser ca. 10'' bei geschlossenem Deckel bissfest dünsten. Darauf achten, dass das Wasser nicht komplett verdunstet.\par\bigskip
	
	\ingredient[\Calc{0.5}{\x}]{l}{Sahne}
	\ingredient[]{}{Salz}
	\ingredient[]{}{Pfeffer}
	\ingredient[]{}{Muskat}
	
	\textbf{Hackfleischbällchen hinzugeben:}\\
	Bissfest gedünstete Kohlrabi mit Sahne versehen. Mit Salz, Pfeffer und Muskat abschmecken. Dann die Hackfleischbällchen (ca. 4cm) formen, in die Soße legen und mit etwas Mehl bestäuben. Dann bei geschlossenem Deckel und mittlerer Hitze ca 10'' ziehen lassen.\par\bigskip
	
	\textbf{Das passt dazu:}\\
	Reis
	
\end{MyRecipe}